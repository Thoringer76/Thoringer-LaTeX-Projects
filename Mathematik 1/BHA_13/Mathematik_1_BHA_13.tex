\documentclass[12pt]{article}

\usepackage{graphicx}			% Use this package to include images
\usepackage{amsmath}			% A library of many standard math expressions
\usepackage{amssymb}
\usepackage[margin=1in]{geometry}% Sets 1in margins. 
\usepackage{fancyhdr}			% Creates headers and footers
\usepackage{enumerate}          %These two package give custom labels to a list
\usepackage[shortlabels]{enumitem}


% Creates the header and footer. You can adjust the look and feel of these here.
\pagestyle{fancy}
\fancyhead[l]{Mathematik 1}
\fancyhead[c]{BHA 13}
\fancyhead[r]{15.1.2024 - 21.1.2024}
\fancyfoot[c]{\thepage}
\renewcommand{\headrulewidth}{0.2pt} %Creates a horizontal line underneath the header
\setlength{\headheight}{15pt} %Sets enough space for the header

\begin{document} %The writing for your homework should all come after this. 

\noindent
\begin{enumerate}[start=1,label={\bfseries Frage \arabic*:},leftmargin=1in]

    \item Betrachtet wird die Funktion $f:\mathbb{R} \to \mathbb{R}$ mit $f(x)=x^{3}-6x+3$. Bestimmen Sie das erste, zweite und dritte Taylorpolynom an der Entwicklungsstelle $x_0=2$.
    \begin{align*}
        f(x)&=x^{3}-6x+3 \\
        f'(x)&=3x^{2}-6 \\
        f''(x)&=6x \\
        f'''(x)&=6 \\
        T_{f,1,2}(x)&=\frac{f(2)}{0!}(x-2)^{0} + \frac{f'(2)}{1!}(x-2)^{1} \\
        &=\frac{-1}{1}1 + \frac{6}{1}(x-2) \\
        &=-1 + 6x - 12 \\
        &=6x - 13 \\
        T_{f,2,2}(x)&=T_{f,1,2}(x) + \frac{f''(x_0)}{2!}(x-x_0)^{2} \\
        &=6x - 13 + (\frac{12}{2!}(x-2)^2) \\
        &=6x - 13 + (6 (x^{2} - 4x + 4)) \\
        &=6x - 13 + (6x^{2} - 24x + 24) \\
        &=6x^{2} - 18x + 11 \\
        T_{f,3,2}(x)&=T_{f,2,2}(x) + \frac{f'''(x_0)}{3!}(x-x_0)^{3} \\
        &=6x^{2} - 18x + 11 + (\frac{6}{3!}(x-2)^{3}) \\
        &=6x^{2} - 18x + 11 + (x^3 - 6x^{2} + 12x - 8) \\
        &=x^3 - 6x + 3
    \end{align*}

    \enlargethispage{-\baselineskip}
    \enlargethispage{-\baselineskip}
    \enlargethispage{-\baselineskip}
    \enlargethispage{-\baselineskip}
    \enlargethispage{-\baselineskip}
    \enlargethispage{-\baselineskip}
    \enlargethispage{-\baselineskip}
    \enlargethispage{-\baselineskip}
    \enlargethispage{-\baselineskip}
    \enlargethispage{-\baselineskip}
    \enlargethispage{-\baselineskip}
    \enlargethispage{-\baselineskip}
    \enlargethispage{-\baselineskip}
    \enlargethispage{-\baselineskip}
    \enlargethispage{-\baselineskip}
    \enlargethispage{-\baselineskip}

    \item Betrachtet wird wie in der Aufgabe zuvor wieder die Funktion $f:\mathbb{R} \to \mathbb{R}$ mit $f(x)=x^{3}-6x+3$. Bestimmen Sie diesmal die Restglieder für das erste, zweite und dritte Taylorpolynom an der Entwicklungsstelle $x_0=2$.
    \begin{align*}
        f''(x)&=6x \\
        f'''(x)&=6 \\
        f^{(4)}(x)&=0 \\
        R_{f,1,2}(x)&=\frac{f''(\xi)}{2!}(x-x_0)^2 \\
        &=\frac{6\xi}{2}(x-2)^2 \\
        &=3\xi (x-2)^2 \\
        R_{f,2,2}(x)&=\frac{f'''(\xi)}{3!}(x-x_0)^3 \\
        &=\frac{6}{6}(x-2)^3 \\
        &=(x-2)^3 \\
        R_{f,3,2}(x)&=\frac{f^{(4)}(\xi)}{4!}(x-x_0)^4 \\
        &=\frac{0}{24}(x-2)^4 \\
        &=0
    \end{align*}

    \item Betrachtet wird die Funktion $f:(0,\infty) \to \mathbb{R}$ mit $f(x)=x^{2} + 2 - x \ln{x}$. Bestimmen Sie das Taylorpolynom $T_{f,2,x_0}(x)$ 2ten Grades an der Entwicklungsstelle $x_0=1$.
    \begin{align*}
        f(x)&=x^{2} + 2 - x \ln{x} \\
        f'(x)&=2x - \ln{x} - 1 \\
        f''(x)&=2 - \frac{1}{x} \\
        T_{f,2,1}(x)&=\frac{f(x_0)}{0!}(x-x_0)^0 + \frac{f'(x_0)}{1!}(x-x_0)^1 + \frac{f''(x_0)}{2!}(x-x_0)^2 \\
        &=\frac{3}{1}(x-1)^0 + \frac{1}{1}(x-1)^1 + \frac{1}{2}(x-1)^2 \\
        &=3 + (x-1) + \frac{1}{2}(x-1)^2
    \end{align*}

    \enlargethispage{-\baselineskip}
    \enlargethispage{-\baselineskip}
    \enlargethispage{-\baselineskip}
    \enlargethispage{-\baselineskip}
    \enlargethispage{-\baselineskip}
        
    \item Betrachtet wird wie in der Aufgabe zuvor wieder die Funktion $f:(0,\infty) \to \mathbb{R}$ mit $f(x)=x^{2} + 2 - x \ln{x}$. Bestimmen Sie für das Taylorpolynom  2ten Grades um die Entwicklungsstelle $x_0=1$ das zugehörige Restglied $R_{f,2,x_0}$
    \begin{align*}
        f'''(x)&=\frac{1}{x^2} \\
        R_{f,2,1}(x)&=\frac{f'''(\xi)}{3!}(x-x_0)^3 \\
        &=\frac{1}{6\xi^2}(x-1)^3 \\
    \end{align*}

    \item Betrachtet wird wie in den Aufgaben zuvor wieder die Funktion $f:(0,\infty) \to \mathbb{R}$ mit $f(x)=x^{2} + 2 - x \ln{x}$. Schätzen Sie den Approximationsfehler für das Taylorpolynoms 2ten Grades für $x \in [\frac{1}{2},1]$ um die Entwicklungsstelle $x_0=1$ ab. (Anmerkung: Abschätzung des Approximationsfehlers heißt Abschätzung des Restglieds.)
    \begin{align*}
        f'''(\frac{1}{2})&=4 \\
        f'''(1)&=1 \\
        R_{f,2,1}(x)&=\frac{1}{6\xi^2}(x-1)^3 \\
        R_{f,2,1}(x)&\le |\frac{1}{6(\frac{1}{2})^2}| \cdot |(\frac{1}{2} - 1)^3| \\
        &\le |\frac{2}{3}| \cdot |(-\frac{1}{8})| \\
        &\le \frac{2}{24} \\
        &\le \frac{1}{12}
    \end{align*}

\end{enumerate}
\end{document}