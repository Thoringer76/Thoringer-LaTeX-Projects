\documentclass[12pt]{article}

\usepackage[dvipsnames]{xcolor}
\usepackage{tikz,lipsum,lmodern}
\usepackage[most]{tcolorbox}
\usepackage{graphicx}			% Use this package to include images
\usepackage{amsmath}			% A library of many standard math expressions
\usepackage{amssymb}
\usepackage[margin=1in]{geometry}% Sets 1in margins. 
\usepackage{fancyhdr}			% Creates headers and footers
\usepackage{enumerate}          %These two package give custom labels to a list
\usepackage[shortlabels]{enumitem}


% Creates the header and footer. You can adjust the look and feel of these here.
\pagestyle{fancy}
\fancyhead[l]{Mathematik 1}
\fancyhead[c]{Formelblatt}
\fancyhead[r]{16.02.2024}
\fancyfoot[c]{\thepage}
\renewcommand{\headrulewidth}{0.2pt} %Creates a horizontal line underneath the header
\setlength{\headheight}{15pt} %Sets enough space for the header

\begin{document} %The writing for your homework should all come after this. 

\noindent

    \begin{tcolorbox}[
    colback=Red!5!white,
    colframe=Red!75!black,
    title={\centering Taylor}]
    \begin{align*}
        T_{f,n,x_0}(x) &: \text{Taylorpolynom von f, n-ten Grades, entwickelt bei} \: x_0 \\
        \text{1. Grad:} \: T_{f,1,x_0}(x) &= f(x_0) + f'(x_0) \cdot (x-x_0) \\
        \text{2. Grad:} \: T_{f,2,x_0}(x) &= T_{f,1,x_0}(x) + \frac{f''(x_0)}{2}(x-x_0)^2 \\
        \text{3. Grad:} \: T_{f,3,x_0}(x) &= T_{f,2,x_0}(x) + \frac{f'''(x_0)}{6}(x-x_0)^3 \\
        \text{4. Grad:} \: T_{f,4,x_0}(x) &= T_{f,3,x_0}(x) + \frac{f^{(4)}(x_0)}{24}(x-x_0)^4 \\
        \text{5. Grad:} \: T_{f,5,x_0}(x) &= T_{f,4,x_0}(x) + \frac{f^{(5)}(x_0)}{120}(x-x_0)^5 \\
        \text{N. Grad:} \: T_{f,n,x_0}(x) &= \sum_{k=0}^n \frac{f^{(k)}(x_0)}{k!}(x-x_0)^k \\
        R_{f,n,x_0}(x) &: \text{Restglied von } \: T_{f,n,x_0}(x) \: \text{, n-ten Grades, entwickelt bei} \: x_0 \\
        \text{1. Grad:} \: R_{f,1,x_0}(x) &= \frac{f''(\xi)}{2}(x-x_0)^2 \\
        \text{2. Grad:} \: R_{f,2,x_0}(x) &= \frac{f'''(\xi)}{6}(x-x_0)^3 \\
        \text{3. Grad:} \: R_{f,3,x_0}(x) &= \frac{f^{(4)}(\xi)}{24}(x-x_0)^4 \\
        \text{4. Grad:} \: R_{f,4,x_0}(x) &= \frac{f^{(5)}(\xi)}{120}(x-x_0)^5 \\
        \text{5. Grad:} \: R_{f,5,x_0}(x) &= \frac{f^{(6)}(\xi)}{720}(x-x_0)^6 \\
        \text{N. Grad:} \: R_{f,n,x_0}(x) &= \frac{f^{(n+1)}(\xi)}{(n+1)!}(x-x_0)^{n+1} \\
        \end{align*}
    \end{tcolorbox}

    \begin{tcolorbox}[
    colback=Orange!5!white,
    colframe=Orange!75!black,
    title={\centering Kurvenintegrale}]
    \begin{align*}
        \text{Länge einer Kurve }(f=1)\text{:} \: \int\limits_{\Vec{r}} 1 \: ds &= \int\limits_{\Vec{r}} ds = \int\limits_a^b \lVert \Dot{\Vec{r}}(t) \rVert \: dt \\
        \text{Kurvenintegral 1. Art:} \: \int\limits_{\Vec{r}} f \: ds &= \int\limits_a^b f(\Vec{r}(t)) \cdot \lVert \Dot{\Vec{r}}(t) \rVert \: dt \\
        \text{Kurvenintegral 2. Art:} \: \int\limits_{\Vec{r}} \Vec{v} \: d\Vec{s} &= \int\limits_a^b \left< \Vec{v}(\Vec{r}(t)) , \Dot{\Vec{r}}(t) \right> \: dt \\
        f(\Vec{r}(t)) &: \text{Vektor in die Funktion einsetzen} \\
        \Vec{v}(\Vec{r}(t)) &: \text{Vektor in die Funktion einsetzen} \\
        \Dot{\Vec{r}}(t) &: \text{Ableitung des Vektors } {\Vec{r}}(t) \\
        \lVert \Dot{\Vec{r}}(t) \rVert &: \text{Betrag des abgeleiteten Vektors} \\
        \left< \Vec{v}(\Vec{r}(t)) , \Dot{\Vec{r}}(t) \right> &: \text{Skalarprodukt von } \Vec{v}(\Vec{r}(t)) \text{ und } \Dot{\Vec{r}}(t)
    \end{align*}
    \end{tcolorbox}

    \begin{tcolorbox}[
    colback=Yellow!5!white,
    colframe=Yellow!75!black,
    title={\centering Lokale Extremstellen}]
    \begin{itemize}
        \item Erste Ableitung der gegebenen Funktion bilden ($f'(x)$)
        \item Erste Ableitung gleich 0 setzen ($f'(x) = 0$) und nach $x$ auflösen
        \item Gefundene $x_0$-Werte in die zweite Ableitung einsetzen ($f''(x_0)$)
        \item Falls $f''(x_0) > 0$: $x_0 \to Minimum$
        \item Falls $f''(x_0) < 0$: $x_0 \to Maximum$
        \item Falls $f''(x_0) = 0$: Vorzeichenwechselkriterium
        \item Falls nach Extrem\textbf{\textit{punkten}} gefragt ist: $x_0$-Werte in Originalfunktion einsetzen und sowohl $x_0$- als auch $y_0$-Werte mit angeben
    \end{itemize}
    \end{tcolorbox}

    \begin{tcolorbox}[
    colback=Green!5!white,
    colframe=Green!75!black,
    title={\centering Grenzwerte, L’Hospital}]
    \begin{align*}
        \text{Ist der Grenzwert} &: \lim_{x \to x_0} \frac{g(x)}{h(x)} \text{ vom Typ } \frac{0}{0} \text{ oder } \frac{\infty}{\infty} \\
        \text{Dann ist} &: \lim_{x \to x_0} \frac{g(x)}{h(x)} \longrightarrow \lim_{x \to x_0} \frac{g'(x)}{h'(x)} \\
        \text{Exponentialfunktion als Grenzwert} &: \lim_{k \to \infty} (1+\frac{x}{a \cdot k})^{b \cdot k} = e^{\frac{b \cdot x}{a}} \\
        \text{Exponentialfunktion als Grenzwert} &: \lim_{k \to \infty} (1-\frac{x}{a \cdot k})^{b \cdot k} = {e^{-\frac{b \cdot x}{a}}}
    \end{align*}
    \end{tcolorbox}
    
    \begin{tcolorbox}[
    colback=Blue!5!white,
    colframe=Blue!75!black,
    title={\centering Reihen, Quotientenkriterium}]
    \begin{align*}
        \text{Geometrische Reihe} &: \sum_{k=0}^{\infty} q^k = \frac{1}{1-q} \: , \: |q|<1 \\
        \text{Quotientenkriterium} &: \sum_{k=0}^{\infty} a_k \text{ konvergiert absolut, wenn: } \lim_{k \to \infty} \frac{|a_{k+1}|}{|a_k|} < 1 \\
        \text{Rechenregel 1} &: (k+1)! = k! \cdot (k+1) \\
        \text{Rechenregel 2} &: n^{k+1} = n^{k} \cdot n
    \end{align*}
    \end{tcolorbox}


















\end{document}