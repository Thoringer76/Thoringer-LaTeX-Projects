\documentclass[12pt]{article}

\usepackage{graphicx}			% Use this package to include images
\usepackage{amsmath}			% A library of many standard math expressions
\usepackage{amssymb}
\usepackage[margin=1in]{geometry}% Sets 1in margins. 
\usepackage{fancyhdr}			% Creates headers and footers
\usepackage{enumerate}          %These two package give custom labels to a list
\usepackage[shortlabels]{enumitem}


% Creates the header and footer. You can adjust the look and feel of these here.
\pagestyle{fancy}
\fancyhead[l]{Mathematik 1}
\fancyhead[c]{BHA 12}
\fancyhead[r]{8.1.2024 - 14.1.2024}
\fancyfoot[c]{\thepage}
\renewcommand{\headrulewidth}{0.2pt} %Creates a horizontal line underneath the header
\setlength{\headheight}{15pt} %Sets enough space for the header

\begin{document} %The writing for your homework should all come after this. 

\noindent
\begin{enumerate}[start=1,label={\bfseries Frage \arabic*:},leftmargin=1in]

    \item Bestimmen Sie die Ableitung der Funktion: $f(x)=\sqrt{x\cdot\sqrt{x}}$
    \begin{align*}
        f(x)&=(x\cdot x^{\frac{1}{2}})^{\frac{1}{2}} \\
        &=x^{\frac{1}{2}} \cdot x^{\frac{1}{4}} \\
        f'(x)&=\frac{1}{2} \cdot x^{-\frac{1}{2}} \cdot x^{\frac{1}{4}} + \frac{1}{4} \cdot x^{\frac{1}{2}} \cdot x^{-\frac{3}{4}} \\
        &=\frac{1}{2} \cdot x^{-\frac{1}{4}} + \frac{1}{4} \cdot x^{-\frac{1}{4}} \\
        &=\frac{1}{2\sqrt[4]{x}} + \frac{1}{4\sqrt[4]{x}} \\
        &=\frac{2}{4\sqrt[4]{x}} + \frac{1}{4\sqrt[4]{x}} \\
        &=\frac{3}{4\sqrt[4]{x}}
    \end{align*}

    \item Bestimmen Sie die Ableitung der folgenden Funktion: $f(x)=x^{5} \cdot 5^{x}$
    \begin{align*}
        f(x)&=x^{5} \cdot e^{x\ln{5}} \\
        f'(x)&=5 \cdot x^{4} \cdot 5^{x} + x^{5} \cdot 5^{x} \cdot \ln{5} \\
        &=5 \cdot x^{4} \cdot 5^{x} + x \cdot x^{4} \cdot 5^{x} \cdot \ln{5} \\
        &=x^{4} \cdot 5^{x} \cdot (5+x\ln{5})
    \end{align*}

    \item Bestimmen Sie die Ableitung der folgenden Funktion: $f(x)=(x \cdot \cos{x})^{x}$
    \begin{align*}
        f(x)&=e^{x \cdot \ln{(x \cdot \cos{x}})} \\
        f'(x)&=(x \cdot \cos{x})^{x} \cdot (x \cdot \ln{(x \cdot \cos{x})})' \\
        &=(x \cdot \cos{x})^{x} \cdot (1 \cdot \ln{(x \cdot \cos{x})} + x \cdot \frac{1}{x \cdot \cos{x}}) \cdot (x \cdot \cos{x})' \\
        &=(x \cdot \cos{x})^{x} \cdot (\ln{(x \cdot \cos{x})} + x \cdot \frac{1}{x \cdot \cos{x}}) \cdot (1 \cdot \cos{x} - x \cdot \sin{x}) \\
        &=(x \cdot \cos{x})^{x} \cdot (\ln{(x \cdot \cos{x})} + \frac{x \cdot \cos{x} - x^{2} \cdot \sin{x}}{x \cdot \cos{x}}) \\
        &=(x \cdot \cos{x})^{x} \cdot (\ln{(x \cdot \cos{x})} + (\frac{x \cos{x}}{x \cos{x}} - \frac{x^{2} \sin{x}}{x \cos{x}})) \\
        &=(x \cdot \cos{x})^{x} \cdot (\ln{(x \cdot \cos{x})} + 1 - x  \cdot \tan{x})
    \end{align*}

    \item Bestimmen Sie die lokalen Maxima und Minima der Funktion $f(x)=x^{3} - 27x$
    \begin{align*}
        f'(x)&=3x^{2} - 27 \\
        f''(x)&=6x \\
        f'(x)&=0 \\
        3x^{2} -27 &= 0 \\
        3x^{2} &= 27 \\
        x^{2} &= 9 \\
        x_{1,2} &= \pm{3} \\
        f''(+3) &= 6 \cdot 3 \\
        &= +18 > 0 \to pMin \quad (x_0 = +3) \\
        f''(-3) &= 6 \cdot (-3) \\
        &= -18 < 0 \to pMax \quad (x_0 = -3)
    \end{align*}

    \item Sei $f:\mathbb{R} \to \mathbb{R}$ eine n-mal differenzierbare Funktion. Geben Sie die n-te Ableitung der Funktion $g(x)$ an: $g(x)=x \cdot f(x)$
    \begin{align*}
        g(x)&=x \cdot f(x) \\
        g^{(1)}(x)&=f(x) + xf^{(1)}(x) \\
        g^{(2)}(x)&=f^{(1)}(x) +  f^{(1)}(x) + xf^{(2)}(x) \\
        g^{(3)}(x)&=f^{(2)}(x) +  f^{(2)}(x) + f^{(2)}(x) + xf^{(3)}(x) \\
        g^{(n)}(x)&=nf^{(n-1)}(x) + xf^{(n)}(x)
    \end{align*}

    \item Bestimmen Sie den folgenden Grenzwert, soweit er existiert, mit Hilfe der Grenzwertregel von Bernoulli und l'Hospital: $\lim\limits_{x \to 0} \frac{\sin{2x}}{3x}$
    \begin{align*}
        f(x)&=\sin{2x} \\
        f'(x)&=2\cos{2x} \\
        g(x)&=3x \\
        g'(x)&=3 \\
        \lim_{x \to 0} \frac{\sin{2x}}{3x} &= \lim_{x \to 0} \frac{2\cos{2x}}{3} \\
        &= \frac{2}{3}
    \end{align*}

    \enlargethispage{-\baselineskip}
    \enlargethispage{-\baselineskip}
    \enlargethispage{-\baselineskip}
    \enlargethispage{-\baselineskip}

    \item Bestimmen Sie den folgenden Grenzwert, soweit er existiert, mit Hilfe der Grenzwertregel von Bernoulli und l'Hospital: $\lim\limits_{x \to \infty} \frac{x^{2}}{e^{x}}$
    \begin{align*}
        f(x)&=x^2 \\
        f'(x)&=2x \\
        f''(x)&=2 \\
        g(x)&=e^x \\
        g'(x)&=e^x \\
        g''(x)&=e^x \\
        \lim_{x \to \infty} \frac{x^2}{e^x} &= \lim_{x \to \infty} \frac{2x}{e^x} \\
        &= \lim_{x \to \infty} \frac{2}{e^x} \\
        &= \frac{2}{\infty} \\
        &= 0
    \end{align*}

\end{enumerate}
\end{document}