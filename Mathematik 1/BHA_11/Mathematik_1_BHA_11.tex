\documentclass[12pt]{article}

\usepackage{graphicx}			% Use this package to include images
\usepackage{amsmath}			% A library of many standard math expressions
\usepackage[margin=1in]{geometry}% Sets 1in margins. 
\usepackage{fancyhdr}			% Creates headers and footers
\usepackage{enumerate}          %These two package give custom labels to a list
\usepackage[shortlabels]{enumitem}


% Creates the header and footer. You can adjust the look and feel of these here.
\pagestyle{fancy}
\fancyhead[l]{Mathematik 1}
\fancyhead[c]{BHA 11}
\fancyhead[r]{18.12.2023 - 7.1.2024}
\fancyfoot[c]{\thepage}
\renewcommand{\headrulewidth}{0.2pt} %Creates a horizontal line underneath the header
\setlength{\headheight}{15pt} %Sets enough space for the header

\begin{document} %The writing for your homework should all come after this. 

\noindent
\begin{enumerate}[start=1,label={\bfseries Frage \arabic*:},leftmargin=1in] 

    \item Bestimmen Sie die Ableitung der folgenden Funktion: $f(x)=\frac{x^3}{3}-{2x^2}+4x-5$ \\
    \textit{Summenregel: $[g(x )+ h(x)]' = g'(x) + h'(x)$} \\
    \begin{align*}
        f'(x)&=(\frac{x^3}{3})'-(2x^2)'+(4x)'-(5)' \\
        &= {x^2}-4x+4\
    \end{align*}

    \item Bestimmen Sie die Ableitung der folgenden Funktion: $f(x)=x\sin{(ax+3)}$ \\
    \textit{Produktregel: $(u \cdot v)' = u' \cdot v + u \cdot v'$} \\
    \textit{Kettenregel: $[f(g(x))]'=f'(g(x)) \cdot g'(x)$}
    \begin{align*}
        f'(x)&=(x)' \cdot \sin({ax+3}) + x \cdot (\sin({ax+3}))' \\
        &= 1 \cdot \sin{(ax+3)} + x \cdot \cos{(ax+3)} \cdot (ax+3)' \\
        &= 1 \cdot \sin{(ax+3)} + x \cdot \cos{(ax+3)} \cdot a \\
        &= \sin{(ax+3)} + ax \cdot \cos{(ax+3)}
    \end{align*}

    \item Bestimmen Sie die Ableitung der folgenden Funktion: $f(x)=\sin^3{x} + \cos^3{x}$ \\
    \textit{Summenregel: $[g(x )+ h(x)]' = g'(x) + h'(x)$} \\
    \textit{Kettenregel: $[f(g(x))]'=f'(g(x)) \cdot g'(x)$}
    \begin{align*}
        f'(x)&=((\sin{x})^3)' + ((\cos{x})^3)' \\
        &= 3(\sin{x})^2 \cdot (\sin{x})' + 3(\cos{x})^2 \cdot (\cos{x})' \\
        &= 3(\sin{x})^2 (\cos{x}) + 3(\cos{x})^2 (-\sin{x}) \\
        &= 3(\sin{x})(\sin{x})(\cos{x}) + 3(\cos{x})(\cos{x})(-\sin{x}) \\
        &= 3(\sin{x})(\sin{x})(\cos{x}) - 3(\cos{x})(\cos{x})(\sin{x}) \\
        &= 3\sin{x}\cos{x} \cdot (\sin{x} - \cos{x})
    \end{align*}

    \item Bestimmen Sie die Ableitung der folgenden Funktion: $f(x)=\frac{\cos{x}}{x^2}$ \\
    \textit{Quotientenregel: $[\frac{g(x)}{h(x)}]' = \frac{g'(x) \cdot h(x) - g(x) \cdot h'(x)}{(h(x))^2}$}
    \begin{align*}
        f'(x)&=\frac{(\cos{x})' \cdot x^2 - \cos{x} \cdot (x^2)'}{(x^2)^2} \\
        &= \frac{-\sin{(x)} \cdot x^2 - \cos{(x)} \cdot 2x}{x^4} \\
        &= \frac{-\sin{(x)} \cdot x - \cos{(x)} \cdot 2}{x^3} \\
        &= \frac{-x\sin{x} - 2\cos{x}}{x^3}
    \end{align*}

    \enlargethispage{-\baselineskip}
    \enlargethispage{-\baselineskip}

    \item Bestimmen Sie die Ableitung der folgenden Funktion: $f(x)=x+\sqrt{x}$ \\
    \textit{Summenregel: $[g(x )+ h(x)]' = g'(x) + h'(x)$} \\
    \begin{align*}
        f'(x)&=(x)' + (x^\frac{1}{2})' \\
        &= 1 + \frac{1}{2} x^{-\frac{1}{2}} \\
        &= 1 + \frac{1}{2x^{\frac{1}{2}}} \\
        &= 1 + \frac{1}{2\sqrt{x}}
    \end{align*}

    \item Bestimmen Sie die Ableitung der folgenden Funktion: $f(x)=x + \sqrt{x^2 + 3}$ \\
    \textit{Summenregel: $[g(x )+ h(x)]' = g'(x) + h'(x)$} \\
    \textit{Kettenregel: $[f(g(x))]'=f'(g(x)) \cdot g'(x)$} \\
    \begin{align*}
        f'(x)&=(x)' + ((x^2 + 3)^\frac{1}{2})' \\
        &= 1 + \frac{1}{2} (x^2 + 3)^{-\frac{1}{2}} \cdot (x^2 + 3)' \\
        &= 1 + \frac{1}{2\sqrt{x^2 + 3}} \cdot 2x \\
        &= 1 + \frac{2x}{2\sqrt{x^2 + 3}} \\
        &= 1 + \frac{x}{\sqrt{x^2 + 3}}
    \end{align*}

    \item Bestimmen Sie die Ableitung der folgenden Funktion: $f(x)=(\sqrt{a} - \sqrt{bx+c})^2$ \\
    \textit{Summenregel: $[g(x )+ h(x)]' = g'(x) + h'(x)$} \\
    \textit{Kettenregel: $[f(g(x))]'=f'(g(x)) \cdot g'(x)$} \\
    \begin{align*}
        f'(x)&=2(\sqrt{a} - \sqrt{bx+c}) \cdot (\sqrt{a} - \sqrt{bx+c})' \\
        &= 2(\sqrt{a} - \sqrt{bx+c}) \cdot (-\frac{1}{2}(bx+c)^{-\frac{1}{2}}) \cdot (bx+c)' \\
        &= 2(\sqrt{a} - \sqrt{bx+c}) \cdot -\frac{1}{2\sqrt{bx+c}} \cdot b \\
        &= -\frac{\sqrt{a}-\sqrt{bx+c}}{\sqrt{bx+c}} \cdot b \\
        &= -(\frac{\sqrt{a}}{\sqrt{bx+c}} - \frac{\sqrt{bx+c}}{\sqrt{bx+c}})b \\
        &= (-\frac{\sqrt{a}}{\sqrt{bx+c}}+1)b \\
        &= b(1-\frac{\sqrt{a}}{\sqrt{bx+c}})
    \end{align*}

    \item Bestimmen Sie die Ableitung der folgenden Funktion: $f(x)=2 - \frac{3}{x} + \frac{4}{x^2} - \frac{5}{x^6}$ \\
    \textit{Summenregel: $[g(x )+ h(x)]' = g'(x) + h'(x)$} \\
    \begin{align*}
        f'(x)&=(2)' - (3x^{-1})' + (4x^{-2}) - (5x^{-6})' \\
        &= 0 - (-1 \cdot 3x^{-2}) + (-2 \cdot 4x^{-3}) - (-6 \cdot 5x^{-7}) \\
        &= 3x^{-2} - 8x^{-3} + 30x^{-7} \\
        &= \frac{3}{x^2} - \frac{8}{x^3} + \frac{30}{x^7}
    \end{align*}

    \item Bestimmen Sie die Ableitung der folgenden Funktion: $f(x)=(e^x \cdot \ln{x})^2$ \\
    \textit{Produktregel: $(u \cdot v)' = u' \cdot v + u \cdot v'$} \\
    \begin{align*}
        f(x)&=e^{2x} \cdot \ln^2{x} \\
        f'(x)&=(e^{2x})' \cdot \ln^2{x} + e^{2x} \cdot (\ln^2{x})' \\
        &= 2e^{2x} \cdot \ln^2{x} + e^{2x} \cdot 2(\ln{x}) \cdot (\ln{x})' \\
        &= 2e^{2x} \cdot \ln^2{x} + e^{2x} \cdot 2(\ln{x}) \cdot \frac{1}{x} \\
        &= 2 (e^{2x})(\ln{x})(\ln{x}) + 2 (e^{2x})(\ln{x})(\frac{1}{x}) \\
        &= 2e^{2x} \cdot \ln{x} \cdot (\ln{x} + \frac{1}{x})
    \end{align*}

    \item Bestimmen Sie $f'(\frac{\sqrt{\pi}}{2})$ für die Funktion: $f(x)=\sqrt{1+(\cos{(x^2))^2}}$ \\
    \textit{Kettenregel: $[f(g(x))]'=f'(g(x)) \cdot g'(x)$} \\
    \begin{align*}
        f'(x)&=\frac{1}{2}(1+(\cos{(x^2)})^2)^{-\frac{1}{2}} \cdot (1+(\cos{(x^2)})^2)' \\
        &= \frac{1}{2}(1+(\cos{(x^2)})^2)^{-\frac{1}{2}} \cdot 2\cos{(x^2)} \cdot (\cos{(x^2)})' \\
        &= \frac{1}{2}(1+(\cos{(x^2)})^2)^{-\frac{1}{2}} \cdot 2\cos{(x^2)} \cdot (-\sin{x^2}) \cdot (x^2)' \\
        &= \frac{1}{2}(1+(\cos{(x^2)})^2)^{-\frac{1}{2}} \cdot 2\cos{(x^2)} \cdot (-\sin{x^2}) \cdot 2x \\
        &= \frac{1}{2\sqrt{1+(\cos{(x^2)})^2}} \cdot 2\cos{(x^2)} \cdot (-\sin{x^2}) \cdot 2x \\  
        &= -\frac{2\cos{(x^2)} \cdot \sin{(x^2)} \cdot 2x}{2\sqrt{1+(\cos{(x^2)})^2}} \\
        &= -\frac{\cos{(x^2)} \cdot \sin{(x^2)} \cdot 2x}{\sqrt{1+(\cos{(x^2)})^2}} \\
    \end{align*}

    \enlargethispage{-\baselineskip}
    
    \textit{Hinweis: $\sqrt{\frac{3}{2}} = \frac{\sqrt{6}}{2} \quad (\sqrt{\frac{3}{2}} = \frac{\sqrt{3}}{\sqrt{2}} = \frac{\sqrt{3} \cdot \sqrt{2}}{\sqrt{2} \cdot \sqrt{2}} = \frac{\sqrt{3 \cdot 2}}{\sqrt{2 \cdot 2}} = \frac{\sqrt{6}}{\sqrt{4}} = \frac{\sqrt{6}}{2})$} \\
    
    \begin{align*}
        f'(\frac{\sqrt{\pi}}{2})&=-\frac{\cos{(\frac{\sqrt{\pi}}{2})^2} \cdot \sin{(\frac{\sqrt{\pi}}{2})^2} \cdot 2(\frac{\sqrt{\pi}}{2})}{\sqrt{1 + (\cos{(\frac{\sqrt{\pi}}{2})^2})^2}} \\
        &=-\frac{\cos{(\frac{\pi}{4})} \cdot \sin{(\frac{\pi}{4})} \cdot \frac{2\sqrt{\pi}}{2}}{\sqrt{1 + (\cos{(\frac{\pi}{4})})^2}} \\
        &= -\frac{\frac{1}{\sqrt{2}} \cdot \frac{1}{\sqrt{2}} \cdot \sqrt{\pi}}{\sqrt{1 + (\frac{1}{\sqrt{2}})^2}} \\
        &= -\frac{\frac{1}{\sqrt{2}} \cdot \frac{1}{\sqrt{2}} \cdot \sqrt{\pi}}{\sqrt{1 + \frac{1}{2}}} \\
        &= -\frac{\frac{\sqrt{\pi}}{2}}{\sqrt{\frac{3}{2}}} \\
        &= -\frac{\frac{\sqrt{\pi}}{2}}{\frac{\sqrt{6}}{2}} \\
        &= -\frac{\sqrt{\pi}}{2} \cdot \frac{2}{\sqrt{6}} \\
        &= -\frac{\sqrt{\pi}}{\sqrt{6}} \\
        &= -\sqrt{\frac{\pi}{6}} \\
    \end{align*}       

\end{enumerate}

\end{document}