\documentclass[12pt]{article}

\usepackage{graphicx}			% Use this package to include images
\usepackage{amsmath}			% A library of many standard math expressions
\usepackage{amssymb}
\usepackage[margin=1in]{geometry}% Sets 1in margins. 
\usepackage{fancyhdr}			% Creates headers and footers
\usepackage{enumerate}          %These two package give custom labels to a list
\usepackage[shortlabels]{enumitem}


% Creates the header and footer. You can adjust the look and feel of these here.
\pagestyle{fancy}
\fancyhead[l]{Mathematik 1}
\fancyhead[c]{BHA 1}
\fancyhead[r]{9.10.2023 - 15.10.2023}
\fancyfoot[c]{\thepage}
\renewcommand{\headrulewidth}{0.2pt} %Creates a horizontal line underneath the header
\setlength{\headheight}{15pt} %Sets enough space for the header

\begin{document} %The writing for your homework should all come after this. 

\noindent
\begin{enumerate}[start=1,label={\bfseries Frage \arabic*:},leftmargin=1in]

    \item Bestimmen Sie alle Lösungen der Gleichung $(x-2)(x-1)(x+1)x=0$
    \begin{align*}
        (x_{1},x_{2},x_{3},x_{4}) = (2,1,-1,0)
    \end{align*}

    \item Bestimmen Sie die Lösung der folgenden Gleichung: $\log_{10}(x-10)=3$
    \begin{align*}
        x = 1010 \text{, da } 10^3 = 1000
    \end{align*}

    \item Bestimmen Sie alle Lösungen der Gleichung $\cos{(3x)}=1$
    \begin{align*}
        x = \frac{k \cdot 2\pi}{3}, k \in \mathbb{Z} \text{, da eine Periode } \frac{2\pi}{3} \text{ lang ist}
    \end{align*}

    \item Vereinfachen Sie den folgenden Ausdruck: \\
          $(\sin{\frac{\pi}{4}} + \sin{\frac{\pi}{3}})(\sin{\frac{\pi}{4}} + \sin{\frac{\pi}{6}})(\cos{\frac{\pi}{4}} - \cos{\frac{\pi}{3}})(\cos{\frac{\pi}{4}} - \cos{\frac{\pi}{6}})$
    \begin{align*}
        -\frac{1}{16}
    \end{align*}

    \item Ziehen Sie die richtigen Antworten in die Felder. (Mehrfachauswahl der \\ 
    Antworten ist möglich. Tipp: Nicht alle vorhandenen Antworten werden gebraucht.) 
    Führen Sie eine Kurvendiskussion der Funktion $f(x)=x^2 - 3x + 2$ durch und vervollständigen Sie die folgenden Aussagen: 
    \begin{itemize}
        \item Der größte Definitionsbereich, der für $f$ möglich ist, sind \textbf{die reellen Zahlen}.
        \item Die Nullstellen von $f$ sind bei \textbf{(1,0) und (2,0)}
        \item $f$ schneidet die y-Achse bei \textbf{(0,2)}
        \item Ist $f$ ungerade (punktsymmetrisch)? \textbf{Nein}
        \item $f$ hat bei \textbf{(1.5, -0.25)} einen \textbf{Tiefpunkt}
    \end{itemize}

    \item Bestimmen Sie eine Koordinatengleichung der Ebene E, die den Punkt \\
    $P(4,5,1)$ enthält und zur Geraden $g:\Vec{x}=\begin{pmatrix} 1 \\ -1 \\ 1 \end{pmatrix} + r \cdot 
    \begin{pmatrix} -1 \\ 2 \\ 0 \end{pmatrix}, r \in \mathbb{R}$, orthogonal ist.
    \begin{align*}
        E : -x_1 + 2x_2 = 6
    \end{align*}

\end{enumerate}
\end{document}