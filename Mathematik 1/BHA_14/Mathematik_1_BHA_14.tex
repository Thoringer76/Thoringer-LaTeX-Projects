\documentclass[12pt]{article}

\usepackage{graphicx}			% Use this package to include images
\usepackage{amsmath}			% A library of many standard math expressions
\usepackage{amssymb}
\usepackage[margin=1in]{geometry}% Sets 1in margins. 
\usepackage{fancyhdr}			% Creates headers and footers
\usepackage{enumerate}          %These two package give custom labels to a list
\usepackage[shortlabels]{enumitem}


% Creates the header and footer. You can adjust the look and feel of these here.
\pagestyle{fancy}
\fancyhead[l]{Mathematik 1}
\fancyhead[c]{BHA 14}
\fancyhead[r]{22.1.2024 - 28.1.2024}
\fancyfoot[c]{\thepage}
\renewcommand{\headrulewidth}{0.2pt} %Creates a horizontal line underneath the header
\setlength{\headheight}{15pt} %Sets enough space for the header

\begin{document} %The writing for your homework should all come after this. 

\noindent
\begin{enumerate}[start=1,label={\bfseries Frage \arabic*:},leftmargin=1in]

    \item Berechnen Sie $\int\limits_0^1 x \: dx - \int\limits_0^1 y \: dy$.
    \begin{align*}
        \int\limits_0^1 x \: dx &= \left[\frac{x^2}{2}\right]_0^1 \\
        &=\left(\frac{1}{2}\right) - \left(\frac{0}{2}\right) \\
        &=\frac{1}{2} \\
        \int\limits_0^1 y \: dy &= \left[\frac{y^2}{2}\right]_0^1 \\
        &=\left(\frac{1}{2}\right) - \left(\frac{0}{2}\right) \\
        &=\frac{1}{2} \\
        \int\limits_0^1 x \: dx - \int\limits_0^1 y \: dy &= \frac{1}{2} - \frac{1}{2} \\
        &=0
    \end{align*}

    \item Die Funktion $f$ sei stetig differenzierbar und von 0 verschieden. Berechnen Sie $\int \frac{f'(x)}{f(x)} \: dx$.
    \begin{align*}
        u&=f(x) \\
        \frac{du}{dx} &= f'(x) \\
        du &= f'(x) dx \\
        \int \frac{f'(x)}{f(x)} \: dx &= \int \frac{du}{u} \\
        &= \int \frac{1}{u} \: du \\
        &= \ln{|u|} + c \\
        &= \ln{|f(x)|} + c
    \end{align*}

    \item Die Funktion $f$ sei stetig differenzierbar. Berechnen Sie $\int f'(x)f(x) \: dx$.
    \begin{align*}
        f(x)&=f(x) \\
        g'(x)&=f'(x) \\
        \int f'(x)f(x) \: dx &= f(x)f(x) - \int f'(x)f(x) \: dx \quad | + \int f'(x)f(x) \: dx \\
        2 \int f'(x)f(x) \: dx &= f(x)f(x) \quad | :2 \\
        \int f'(x)f(x) \: dx &= \frac{(f(x))^2}{2} + c
    \end{align*}

    \item Wie lauten mögliche Stammfunktionen von $f:(0,\infty) \to \mathbb{R}$ mit $f(x)=\ln{(x)}$?
    \textit{Hinweis: $(\sin{(x)})^2 + (\cos{(x)})^2 = 1$}
    \begin{align*}
        f(x)&=\ln{x} \\
        g'(x)&=1 \\
        \int 1 \cdot \ln{x} \: dx &= x\ln{x} - \int \frac{1}{x} x \: dx \\
        &= x\ln{x} - \int 1 \: dx \\
        &= x \ln{x} - x + c \\
        c &= (\sin{(x)})^2 + (\cos{(x)})^2 \\
    \end{align*}

    \item Die Funktionsgraphen von $f:\mathbb{R} \to \mathbb{R}$ mit $f(x)=3-x^2$ und $g:\mathbb{R} \to \mathbb{R}$ mit $g(x)=4x-2$ schließen eine Fläche ein. Wie viele Flächeneinheiten werden eingeschlossen?
    \begin{align*}
        f(x)-g(x) &=-x^2 -4x +5 \\
        0&=x^2 + 4x -5 \\
        x_{1,2} &= -\frac{4}{2} \pm \sqrt{\left(\frac{4}{2}\right)^2 + 5} \\
        x_1 &= -5 \\
        x_2 &= 1 \\
        \int\limits_{-5}^{1} \left(-x^2 -4x +5\right) \: dx &= \left[-\frac{x^3}{3} - 2x^{2} + 5x \right]_{-5}^{1}\\
        &=\left(-\frac{1}{3} - 2 + 5\right) - \left(-\frac{-125}{3} - 50 - 25\right) \\
        &=\left(\frac{8}{3}\right) - \left(-\frac{100}{3}\right) \\
        &=36
    \end{align*}
\end{enumerate}
\end{document}