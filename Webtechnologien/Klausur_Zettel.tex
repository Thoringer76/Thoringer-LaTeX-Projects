\documentclass[10pt]{article}

\usepackage[dvipsnames]{xcolor}
\usepackage{tikz,lipsum,lmodern}
\usepackage[most]{tcolorbox}
\usepackage{graphicx}
\usepackage{amsmath}
\usepackage{amssymb}
\usepackage[margin=1in]{geometry}
\usepackage{fancyhdr}
\usepackage{enumerate}
\usepackage[shortlabels]{enumitem}
\usepackage{listings}

\pagestyle{fancy}
\fancyhead[l]{Webtechnologien}
\fancyhead[c]{Typescript}
\fancyhead[r]{28.02.2024}
\fancyfoot[c]{\thepage}
\renewcommand{\headrulewidth}{0.2pt}
\setlength{\headheight}{15pt}

\begin{document} 

\noindent

    \begin{tcolorbox}[
    colback=Red!5!white,
    colframe=Red!75!black,
    title={\centering Allgemeines}]
    \begin{itemize}
        \item Syntaktische Obermenge von JavaScript
        \item Fügt statische Typisierung hinzu
        \item Ermöglicht Angabe der Datentypen
        \item Typüberprüfung erfolgt vorab während Kompilierung
        \item JavaScript:
        \begin{itemize}
            \item Sprache mit loser Typisierung
            \item Typ von Variablen und Funktionsparametern wird durch ihre Verwendung festgelegt
        \end{itemize}
        \item TypeScript Compiler wandet TypeScript Code in JavaScript Code um
        \item Installation des Compilers mittels node.js Package Manager (npm)
        \item Ausführung des Compilers mittels npx
        \item TypeScript-Compiler wird durch Datei tsconfig.json konfiguriert
        \item Initiierung durch das Kommando npx tsc -init
        \item Festlegung z.B. von Eingabe- und Ausgabeverzeichnissen
    \end{itemize}
    \begin{lstlisting}
        {
            "include": ["src"],
            "compilerOptions": {
                "outDir": "./build"
            }
        }
    \end{lstlisting}
    \end{tcolorbox}

    \begin{tcolorbox}[
    colback=Orange!5!white,
    colframe=Orange!75!black,
    title={\centering Typen}]
    \begin{itemize}
        \item Typen in TypeScript können explizit oder implizit festgelegt werden
        \item Explizite Festlegung: Variablenname gefolgt von Doppelpunkt und Typ
        \item Typen: \textbf{number, string, boolean}
        \item Bei impliziter Typ-Festlegung schließt TypeScript den Typ auf Basis der ersten Zuweisung
        \item In beiden Fällen ist der Typ der Variablen hinterher String (Zeichenkette)
    \end{itemize}
    \begin{lstlisting}
        // explizite Typ-Festlegung
        let vorname: string = "Olaf";

        // implizite Typ-Festlegung
        let vorname2 = "Friedrich";
    \end{lstlisting}
    \begin{itemize}
        \item Spezielle Typen: \textbf{any} und \textbf{unknown}
        \item Wenn TypeScript auf einen Typ nicht schließen kann, wird die Typisierung hierfür außer Kraft gesetzt (Typ: \textbf{any})
        \item Verhindert keine falsche Nutzung
        \item Besser: Typ explizit auf unbekannt (\textbf{unknown}) setzen und vor der Verwendung ggf. auf korrekten Typ casten
    \end{itemize}
    \begin{lstlisting}
        let ergebnis: any = 33;
        ergebnis = "Christian"; // kein Fehler!
        if (ergebnis == false)  // kein Fehler!
            ergebnis = true;

        let resultat: unknown = 11;
    \end{lstlisting}
    \begin{itemize}
        \item Fehler bei Typ-Wechsel: Fehlermeldung, wenn Typ eines zugewiesenen Wertes und zuvor (explizit oder implizit) festgelegter Typ nicht übereinstimmen!
    \end{itemize}
    \begin{lstlisting}
        // explizite Typ-Festlegung
        let vorname: string = "Olaf";
        vorname = 3.14;

        // implizite Typ-Festlegung
        let vorname2 = "Friedrich";
        vorname2 = true;
    \end{lstlisting}
    \end{tcolorbox}

    \begin{tcolorbox}[
    colback=Yellow!5!white,
    colframe=Yellow!75!black,
    title={\centering Arrays}]
    \begin{itemize}
        \item Zusätzliche eckige Klammern nach dem Typ
        \item Alle Array-Elemente haben den gleichen Typ!
        \item Auch bei Arrays kann die Typ-Zuweisung implizit erfolgen
        \item Arrays, deren Inhalte nicht verändert werden können, verwenden zusätzlich das Attribut \textbf{readonly}
    \end{itemize}
    \begin{lstlisting}
        const namen: string[] = [];
        // JS: const name = [];
        name[0] = "Robert";
        name[1] = 42;

        const vornamen: readonly string[] = [];
        vornamen[0] = "Annalena";
    \end{lstlisting}
    \end{tcolorbox}

    \begin{tcolorbox}[
    colback=Green!5!white,
    colframe=Green!75!black,
    title={\centering Tupel}]
    \begin{itemize}
        \item Typisiertes Array mit einer vordefinierten Länge und festgelegten Typen
        \item Für jeden Index wird der Typ vorab explizit festgelegt
        \item Für höhere Indexwerte ist der Typ immer \textbf{any}
        \item Höhere Indexwerte ausschließen mit \textbf{readonly}
    \end{itemize}
    \begin{lstlisting}
        let meinTupel: [string, number, boolean];
        meinTupel = ['Hubertus', 51, true];
        meinTupel[2] = 12; // Fehler!
        meinTupel[4] = "Boris"; // kein Fehler!
        
        let meinAnderesTupel: readonly [string, number];
        meinAnderesTupel = ['Nancy', 53];
        meinAnderesTupel[2] = true; // Fehler!
    \end{lstlisting}
    \end{tcolorbox}





\end{document}
