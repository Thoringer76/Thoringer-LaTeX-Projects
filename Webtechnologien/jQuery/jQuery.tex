\documentclass[10pt]{article}

\usepackage[dvipsnames]{xcolor}
\usepackage{tikz,lipsum,lmodern}
\usepackage[most]{tcolorbox}
\usepackage{graphicx}
\usepackage{amsmath}
\usepackage{amssymb}
\usepackage[margin=1in]{geometry}
\usepackage{fancyhdr}
\usepackage{enumerate}
\usepackage[shortlabels]{enumitem}
\usepackage{listings}

\pagestyle{fancy}
\fancyhead[l]{Webtechnologien}
\fancyhead[c]{jQuery}
\fancyhead[r]{28.02.2024}
\fancyfoot[c]{\thepage}
\renewcommand{\headrulewidth}{0.2pt}
\setlength{\headheight}{15pt}

\begin{document} 

\noindent

    \begin{tcolorbox}[
    colback=Red!5!white,
    colframe=Red!75!black,
    title={\centering Allgemeines}]
    \begin{itemize}
        \item Externe JavaScript-Bibliothek
        \item Vereinfacht die Nutzung von JavaScript für Webseiten
        \item Erlaubt:
        \begin{itemize}
            \item Einfachen Zugriff auf HTML-Elemente
            \item Einfache Ausführung von Aktionen auf diese Elemente
        \end{itemize}
        \item Bibliothek muss entweder auf lokalem Server installiert werden oder extern verlinkt werden
        \item Download von: jQuery.com
        \item Einbindung im Kopfbereich
        \item Alternativ: Google Content Delivery Network
        \item Google CDN schneller, da ggf. schon in Cache, aber problematisch wegen DGSVO (vgl. Google Fonts)
    \end{itemize}
    \begin{lstlisting}
        \textless head\textgreater 
            \textless script src="jquery-3.7.1.min.js"\textgreater 
                \textless /script\textgreater 
        \textless /head\textgreater 

        // from CDN
        \textless head\textgreater 
            \textless script src="https://ajax.googleapis.com/ajax/
                libs/jquery/3.7.1/jquery.min.js"\textgreater 
                    \textless /script\textgreater 
        \textless /head\textgreater 
    \end{lstlisting}
    \end{tcolorbox}

    \begin{tcolorbox}[
    colback=Red!5!white,
    colframe=Red!75!black,
    title={\centering Syntax}]
    \begin{lstlisting}
        $(selector).action()
    \end{lstlisting}
    \begin{itemize}
        \item Alle jQuery-Kommandos beginnen mit dem Dollar-Zeichen
    \end{itemize}
    \begin{lstlisting}
        $(this).hide(); // versteckt das aktuelle Element

        $("h2").hide(); // versteckt alle \textless h2\textgreater 
            -Elemente

        $(".myClass").hide(); // versteckt alle Elemente der
                              // der Klasse "myClass"

        $("#myID").hide(); // versteckt das Element mit
                           // der ID "myID"
    \end{lstlisting}
    \end{tcolorbox}

    \begin{tcolorbox}[
    colback=Red!5!white,
    colframe=Red!75!black,
    title={\centering Selektoren}]
    \begin{itemize}
        \item Selektion aller Elemente mit dem Tag \textbf{tagname}:
    \end{itemize}
    \begin{lstlisting}
        $("tagname")
    \end{lstlisting}
    \begin{itemize}
        \item Selektion des Elements mit der ID \textbf{id}:
    \end{itemize}
    \begin{lstlisting}
        $("#id")
    \end{lstlisting}
    \begin{itemize}
        \item Selektion aller Elemente der Klasse \textbf{class}:
    \end{itemize}
    \begin{lstlisting}
        $(".class")
    \end{lstlisting}
    \begin{itemize}
        \item Selektion aller Elemente mit dem Tag \textbf{tagname}, deren Attribut \textbf{attribute} den Wert \textbf{value} aufweist
    \end{itemize}
    \begin{lstlisting}
        $("tagname[attribute='value']'')
    \end{lstlisting}
    \begin{itemize}
        \item Mehrere Alternative können durch Kommata getrennt kombiniert werden
    \end{itemize}
    \begin{lstlisting}
        $("#myId,.myClass,.my2ndClass")
    \end{lstlisting}
    \end{tcolorbox}

    \begin{tcolorbox}[
    colback=Red!5!white,
    colframe=Red!75!black,
    title={\centering Ereignisse/Aktionen}]
    \begin{itemize}
        \item Ereignisse lösen Aktionen aus
        \item z.B. Maus-Klick, Mauszeiger auf ein Element bewegen, Tastatureingaben, Scrollen, etc.
        \item Entsprechende Events werden nach dem Selektor mit dem Punktoperator abgetrennt
        \item Damit auch eine entsprechende Aktion erfolgt, muss diese als Funktionsaufruf innerhalb der Ereignisverarbeitung definiert werden
        \item Funktionen können auch direkt inline definiert werden
    \end{itemize}
    \begin{lstlisting}
        $("h1").mouseenter();

        // ---

        $("h1").mouseenter(function(){
            alert("Beruehrt");
        })
    \end{lstlisting}
    \begin{itemize}
        \item Mausereignisse:
        \begin{itemize}
            \item click
            \item dblclick
            \item mouseenter
            \item mouseleave
        \end{itemize}
        \item Tastaturereignisse
        \begin{itemize}
            \item keypress
            \item keydown
            \item keyup
        \end{itemize}
        \item Formularereignisse
        \begin{itemize}
            \item submit
            \item change
            \item focus
            \item blur
        \end{itemize}
        \item Seitenereignisse
        \begin{itemize}
            \item load
            \item resize
            \item scroll 
            \item unload
        \end{itemize}
    \end{itemize}
    \end{tcolorbox}

    \begin{tcolorbox}[
    colback=Red!5!white,
    colframe=Red!75!black,
    title={\centering Ready-Funktion}]
    \begin{lstlisting}
        $(document).ready(function(){
            $("button").click(function(){
                $("#element2hide").hide();
            });
        });
    \end{lstlisting}
    \begin{itemize}
        \item Es ist sinnvoll, jQuery-Anweisungen innerhalb eines Ready- Events auszuführen:
        \begin{itemize}
            \item Verhindert die Ausführung vor Aufbau der kompletten Webseite
            \item Hier: Auszublendendes Element könnte u.U. noch gar nicht angezeigt werden
        \end{itemize}
    \end{itemize}
    \end{tcolorbox}

    \begin{tcolorbox}[
    colback=Red!5!white,
    colframe=Red!75!black,
    title={\centering Selektorenzusätze}]
        \textbf{Selektorzusatz  Beispiel  Beschreibung} \\
        :first  \$("p:first")  Erstes Absatzelement \\
        :last  \$("p:last")  Letztes Absatzelement \\
        :even  \$("tr:even")  Gerade Tabellenreihen \\
        :odd  \$("tr:odd")  Ungerade Tabellenreihen \\
        parent \textgreater  child  \$("div \textgreater  p")  \textless p\textgreater -Elemente, die direkte Kindelemente eines \textless div\textgreater -Containers sind \\
        parent descendant  \$("div p")  \textless p\textgreater -Elemente, die Teil eines \textless div\textgreater -Containers sind \\
        parent + next  \$("div + p")  \textless p\textgreater -Elemente, die unmittelbar vor oder hinter einem \textless div\textgreater -Container stehen \\
        element ~ siblings  \$("div ~ p")  \textless p\textgreater -Elemente, die auf gleicher Stufe wie ein \textless div\textgreater -Container stehen \\
        :eq(index)  \$("li ol:eq(2)")  2. Element in einer nummerierten Liste \\
        :gt(value)  \$("li ul:gt(5)")  Alle Elemente in einer unnummerierten Liste ab dem 7.(!) Element \\
        :lt(value)  \$("li ol:lt(4)")  Die Elemente 1-4 (Indizes 0-3) einer nummerierten Liste \\
        :not(select or)  \$("input:not(empty"))  Alle nicht leeren Input-Elemente \\
        :focus  \$(":focus")  Element, was aktuell den Fokus hat \\
        :contains(text)  \$(":contains('TEST')")  Alle Elemente, die das Wort “TEST“ enthalten \\
        :has(selector)  \$("div:has(ul)")  Alle \textless div\textgreater -Container, die ein unnummeriertes Listenelement enthalten \\
        :empty  \$("p:emtpy")  Alle \textless p\textgreater -Elemente, die leer sind \\
        :hidden  \$("p:hidden")  Alle \textless p\textgreater -Elemente, die nicht sichtbar sind \\
        :visible  \$("h1:visible")  Alle \textless h1\textgreater -Elemente, die sichtbar sind \\
        :input  \$(":input")  Alle Input-Elemente \\
        :button  \$(":button")  Alle Button-Elemente \\
        :image  \$(“image")  Alle Bild-Elemente \\
        :enabled  \$(":enabled")  Alle aktiven Input-Elemente \\
        :disabled  \$(":disabled")  Alle inaktiven Input-Elemente \\
        :selected  \$(":selected")  Alle ausgewählten Input-Elemente \\
        :checked  \$(":checked")  Alle angeklickten Check-Boxen
    \end{tcolorbox}

    \begin{tcolorbox}[
    colback=Red!5!white,
    colframe=Red!75!black,
    title={\centering Beispiel}]
    \begin{lstlisting}
        <!DOCTYPE html>
        <html> 
        <head> 
            <script src="https://ajax.googleapis.com/ajax/libs/jquery\
                /3.7.1/jquery.min.js"></script>
            <script>
                $(document).ready(function(){
                    $("#hide").click(function(){
                        $("img").hide();
                    });
                    $("#show").click(function(){
                        $("img").show();
                    });
                });
            </script>
            </head>
            <body>
                <img src="Ilmenau.jpg" width="400" height="300"/>
                <button id="hide">Hide</button>
                <button id="show">Show</button>
            </body>
        </html>
    \end{lstlisting}
    \end{tcolorbox}

    \begin{tcolorbox}[
    colback=Red!5!white,
    colframe=Red!75!black,
    title={\centering Blenden (Fading)}]
    \begin{itemize}
        \item Einblenden ausgewählter Elemente mit der Geschwindigkeit speed ("slow", "fast", Zeit in ms)
        \begin{itemize}
            \item Anschließend Aufruf der Callback—Funktion
            \item Alle Parameter optional
        \end{itemize}
        \$(selector).fadeIn(speed, callback);
    \end{itemize}
    \begin{itemize}
        \item Ausblenden ausgewählter Elemente: \\ \$(selector).fadeOut(speed, callback);
    \end{itemize}
    \begin{itemize}
        \item Ein- oder Ausblenden je nach Zustand: \\ \$(selector).fadeToggle(speed, callback);
    \end{itemize}
    \begin{itemize}
        \item Ein-/ Ausblenden bis zu einem festgelegten Transparenzgrad \\ \$(selector).fadeTo(speed, opacity, callback);
    \end{itemize}
    \end{tcolorbox}

    \begin{tcolorbox}[
    colback=Red!5!white,
    colframe=Red!75!black,
    title={\centering Klappen (Sliding)}]
    \begin{itemize}
        \item Ausklappen ausgewählter Elemente mit der Geschwindigkeit speed ("slow", "fast", Zeit in ms)
        \begin{itemize}
            \item Anschließend Aufruf der Callback—Funktion
            \item Alle Parameter optional
        \end{itemize}
        \$(selector).slideDown(speed, callback);
    \end{itemize}
    \begin{itemize}
        \item Einklappen ausgewählter Elemente: \\ \$(selector).slideUp(speed, callback);
    \end{itemize}
    \begin{itemize}
        \item Ein- oder Ausklappen je nach Zustand: \\ \$(selector).slideToggle(speed, callback);
    \end{itemize}
    \end{tcolorbox}

    \begin{tcolorbox}[
    colback=Red!5!white,
    colframe=Red!75!black,
    title={\centering Animationen}]
    \begin{itemize}
        \item Einfache Animationen \\ \$(selector).animate({parameter},speed, callback);
    \end{itemize}
    \begin{itemize}
        \item Positionsparameter für Abstand von Seitenbegrenzung (left, right, top, bottom)
        \item left:'250px' // Verschieben zur Pos. 250px von linkem Seitenr.
        \begin{itemize}
            \item Mehrere Parameter werden durch Kommata voneinander getrennt
            \item Standardmäßig haben alle HTML-Elemente eine statische Position, die nicht verschoben werden kann.
            \item Um die Position zu verändern, müssen Sie zuerst die CSS-Eigenschaft position des Elements auf relativ, fest oder absolut einstellen!
            \item Alle CSS-Styles als Parameter möglich!
        \end{itemize}
        \item Beispiele: opacity, width, height, fontSize, etc.
        \item Relative Angaben durch vorangestelltes += möglich, z.B.: '+=50px'
    \end{itemize}
    \end{tcolorbox}

    \begin{tcolorbox}[
    colback=Red!5!white,
    colframe=Red!75!black,
    title={\centering Beispiel}]
    \begin{lstlisting}
        <!DOCTYPE html>
        <html>
            <head>
                <script src="https://ajax.googleapis.com/ajax/libs/
                jquery/3.7.1/jquery.min.js"></script>
                <script>
                    $(document).ready(function(){
                        $("button").click(function(){
                            $("div").animate({
                            left: '+=25px',
                            opacity: '0.5',
                            height: '+=15px',
                            fontSize: '+=4px',
                            width: '100px'
                            });
                        });
                    });
                </script>
            </head>
            <body>
                <button>Run Animation</button>
                <div style="background:#8080F0;
                    height:50px; width:50px; position:absolute;">TEST
                        </div>
            </body>
        </html>
    \end{lstlisting}
    \end{tcolorbox}

    \begin{tcolorbox}[
    colback=Red!5!white,
    colframe=Red!75!black,
    title={\centering Inhalte und Attribute auslesen}]
    \begin{itemize}
        \item Zugriff auf Elementinhalte \\ \$(selector).text();
        \item Elementinhalte inklusive HTML-Tags: \\ \$(selector).html();
        \item Werte von Formularfeldern: \\ \$(selector).val();
        \item Werte von Attributen: \\ \$(selector).attr(attrName);
    \end{itemize}
    \end{tcolorbox}

    \begin{tcolorbox}[
    colback=Red!5!white,
    colframe=Red!75!black,
    title={\centering Inhalte und Attribute setzen}]
    \begin{itemize}
        \item Zugriff auf Elementinhalte \\ \$(selector).text(value);
        \item Elementinhalte inklusive HTML-Tags: \\ \$(selector).html(value);
        \item Werte von Formularfeldern: \\ \$(selector).val(value);
        \item Werte von Attributen: \\ \$(selector).attr(attrName, value);
    \end{itemize}
    \end{tcolorbox}

    \begin{tcolorbox}[
    colback=Red!5!white,
    colframe=Red!75!black,
    title={\centering Elemente löschen oder leeren}]
    \begin{itemize}
        \item Elemente entfernen \\ \$(selector).remove()
        \item Elementinhalte entfernen: \\ \$(selector).empty();
    \end{itemize}
    \end{tcolorbox}

    \begin{tcolorbox}[
    colback=Red!5!white,
    colframe=Red!75!black,
    title={\centering Elemente einfügen}]
    \begin{itemize}
        \item Inhalte am Ende des selektierten Elements einhängen \\ \$(selector).append(newContent)
        \item Inhalte am Anfang des selektierten Elements einhängen: \\ \$(selector).prepend(newContent);
        \item Inhalte nach dem selektierten Elements einfügen: \\ \$(selector).after(newContent);
        \item Inhalte vor dem selektierten Elements einfügen : \\ \$(selector).before(newContent);
        \item Mehrere Elemente können durch Kommata voneinander getrennt gleichzeitig eingefügt werden
    \end{itemize}
    \end{tcolorbox}

\end{document}