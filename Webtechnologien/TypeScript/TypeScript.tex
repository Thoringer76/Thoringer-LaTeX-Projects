\documentclass[10pt]{article}

\usepackage[dvipsnames]{xcolor}
\usepackage{tikz,lipsum,lmodern}
\usepackage[most]{tcolorbox}
\usepackage{graphicx}
\usepackage{amsmath}
\usepackage{amssymb}
\usepackage[margin=1in]{geometry}
\usepackage{fancyhdr}
\usepackage{enumerate}
\usepackage[shortlabels]{enumitem}
\usepackage{listings}

\pagestyle{fancy}
\fancyhead[l]{Webtechnologien}
\fancyhead[c]{Typescript}
\fancyhead[r]{28.02.2024}
\fancyfoot[c]{\thepage}
\renewcommand{\headrulewidth}{0.2pt}
\setlength{\headheight}{15pt}

\begin{document} 

\noindent

    \begin{tcolorbox}[
    colback=Red!5!white,
    colframe=Red!75!black,
    title={\centering Allgemeines}]
    \begin{itemize}
        \item Syntaktische Obermenge von JavaScript
        \item Fügt statische Typisierung hinzu
        \item Ermöglicht Angabe der Datentypen
        \item Typüberprüfung erfolgt vorab während Kompilierung
        \item JavaScript:
        \begin{itemize}
            \item Sprache mit loser Typisierung
            \item Typ von Variablen und Funktionsparametern wird durch ihre Verwendung festgelegt
        \end{itemize}
        \item TypeScript Compiler wandet TypeScript Code in JavaScript Code um
        \item Installation des Compilers mittels node.js Package Manager (npm)
        \item Ausführung des Compilers mittels npx
        \item TypeScript-Compiler wird durch Datei tsconfig.json konfiguriert
        \item Initiierung durch das Kommando npx tsc -init
        \item Festlegung z.B. von Eingabe- und Ausgabeverzeichnissen
    \end{itemize}
    \begin{lstlisting}
        {
            "include": ["src"],
            "compilerOptions": {
                "outDir": "./build"
            }
        }
    \end{lstlisting}
    \end{tcolorbox}

    \begin{tcolorbox}[
    colback=Orange!5!white,
    colframe=Orange!75!black,
    title={\centering Typen}]
    \begin{itemize}
        \item Typen in TypeScript können explizit oder implizit festgelegt werden
        \item Explizite Festlegung: Variablenname gefolgt von Doppelpunkt und Typ
        \item Typen: \textbf{number, string, boolean}
        \item Bei impliziter Typ-Festlegung schließt TypeScript den Typ auf Basis der ersten Zuweisung
        \item In beiden Fällen ist der Typ der Variablen hinterher String (Zeichenkette)
    \end{itemize}
    \begin{lstlisting}
        // explizite Typ-Festlegung
        let vorname: string = "Olaf";

        // implizite Typ-Festlegung
        let vorname2 = "Friedrich";
    \end{lstlisting}
    \begin{itemize}
        \item Spezielle Typen: \textbf{any} und \textbf{unknown}
        \item Wenn TypeScript auf einen Typ nicht schließen kann, wird die Typisierung hierfür außer Kraft gesetzt (Typ: \textbf{any})
        \item Verhindert keine falsche Nutzung
        \item Besser: Typ explizit auf unbekannt (\textbf{unknown}) setzen und vor der Verwendung ggf. auf korrekten Typ casten
    \end{itemize}
    \begin{lstlisting}
        let ergebnis: any = 33;
        ergebnis = "Christian"; // kein Fehler!
        if (ergebnis == false)  // kein Fehler!
            ergebnis = true;

        let resultat: unknown = 11;
    \end{lstlisting}
    \begin{itemize}
        \item Fehler bei Typ-Wechsel: Fehlermeldung, wenn Typ eines zugewiesenen Wertes und zuvor (explizit oder implizit) festgelegter Typ nicht übereinstimmen!
    \end{itemize}
    \begin{lstlisting}
        // explizite Typ-Festlegung
        let vorname: string = "Olaf";
        vorname = 3.14;

        // implizite Typ-Festlegung
        let vorname2 = "Friedrich";
        vorname2 = true;
    \end{lstlisting}
    \end{tcolorbox}

    \begin{tcolorbox}[
    colback=Yellow!5!white,
    colframe=Yellow!75!black,
    title={\centering Arrays}]
    \begin{itemize}
        \item Zusätzliche eckige Klammern nach dem Typ
        \item Alle Array-Elemente haben den gleichen Typ!
        \item Auch bei Arrays kann die Typ-Zuweisung implizit erfolgen
        \item Arrays, deren Inhalte nicht verändert werden können, verwenden zusätzlich das Attribut \textbf{readonly}
    \end{itemize}
    \begin{lstlisting}
        const namen: string[] = [];
        // JS: const name = [];
        name[0] = "Robert";
        name[1] = 42;

        const vornamen: readonly string[] = [];
        vornamen[0] = "Annalena";
    \end{lstlisting}
    \end{tcolorbox}

    \begin{tcolorbox}[
    colback=Green!5!white,
    colframe=Green!75!black,
    title={\centering Tupel}]
    \begin{itemize}
        \item Typisiertes Array mit einer vordefinierten Länge und festgelegten Typen
        \item Für jeden Index wird der Typ vorab explizit festgelegt
        \item Für höhere Indexwerte ist der Typ immer \textbf{any}
        \item Höhere Indexwerte ausschließen mit \textbf{readonly}
    \end{itemize}
    \begin{lstlisting}
        let meinTupel: [string, number, boolean];
        meinTupel = ['Hubertus', 51, true];
        meinTupel[2] = 12; // Fehler!
        meinTupel[4] = "Boris"; // kein Fehler!
        
        let meinAnderesTupel: readonly [string, number];
        meinAnderesTupel = ['Nancy', 53];
        meinAnderesTupel[2] = true; // Fehler!
    \end{lstlisting}
    \end{tcolorbox}

    \begin{tcolorbox}[
    colback=Blue!5!white,
    colframe=Blue!75!black,
    title={\centering Objekte}]
    \begin{itemize}
        \item Typ für jede Eigenschaft wird individuell festgelegt
        \item Zuweisungen falschen Typs führen zu Fehlermeldungen
    \end{itemize}
    \begin{lstlisting}
        const kaffeeautomat: {marke: string, modell: string,
        seriennr: number} = {
            marke: "Jura",
            modell: 'F9',
            seriennr: 3318482
        };
    \end{lstlisting}
    \begin{itemize}
        \item Alle definierten Felder müssen initialisiert werden!
        \item Optionale Felder können entsprechend mit \textbf{?} definiert werden
    \end{itemize}
    \begin{lstlisting}
        const kaffeeautomat: {marke: string, modell: string,
        seriennr: number} = {
            marke: "Jura",
            modell: 'F9'
        };

        const kaffeeautomat: {marke: string, modell: string,
        seriennr?: number} = {
            marke: "Jura",
            modell: 'F9'
        };
    \end{lstlisting}
    \end{tcolorbox}

    \begin{tcolorbox}[
    colback=Purple!5!white,
    colframe=Purple!75!black,
    title={\centering Aufzählungen}]
    \begin{itemize}
        \item Repräsentieren Gruppen von Konstanten
        \item Aufzählungen können numerisch sein oder auf Zeichenketten beruhen
        \item Standardmäßig hat die erste Konstante den Wert 0, die zweite 1, die dritte 2, etc.
        \item Es ist möglich, einzelne Werte anzupassen, nachfolgende Werte sind jeweils 1 höher
        \item Alle Konstanten müssen jedoch unterschiedliche Werte haben
    \end{itemize}
    \begin{lstlisting}
        enum richtung {
            VORWAERTS, RUECKWAERTS, LINKS, RECHTS, HOCH, RUNTER
        };
        // VORWAERTS = 0 ... RUNTER = 5

        enum farbe { ROT = 1, GRUEN, GELB, BLAU, ORANGE = 10
        };
        //GRUEN = 2 ... BLAU = 4
    \end{lstlisting}
    \begin{itemize}
        \item Alternativ können Aufzählungen Zeichenketten enthalten
        \item Zugriff auf einzelne Konstanten mittels Punktoperator
    \end{itemize}
    \begin{lstlisting}
        enum wochentag {
            MONTAG = 'Montag',
            DIENSTAG = 'Dienstag',
            MITTWOCH = 'Mittwoch',
            DONNERSTAG = 'Donnerstag',
            FREITAG = 'Freitag',
            SAMSTAG = 'Samstag',
            SONNTAG = 'Sonntag'
        };

        foo(wochentag.FREITAG); // Parameter "Freitag"
    \end{lstlisting}
    \end{tcolorbox}

    \begin{tcolorbox}[
    colback=Blue!5!white,
    colframe=Blue!75!black,
    title={\centering Typen Aliase}]
    \begin{itemize}
        \item Für die weitere Verwendung kann man Typen benennen und anschließend wie Basistypen verwenden
        \item Insbesondere sinnvoll für komplexere Typen (Arrays, Tupel, Objekte, Aufzählungen)
    \end{itemize}
    \begin{lstlisting}
        type Person = [string, string, number];
        type Alter = number;

        let km: Person = ["Mustermann", "Klaus", 40];
        let alterKM: Alter = 40;
    \end{lstlisting}
    \end{tcolorbox}

    \begin{tcolorbox}[
    colback=Green!5!white,
    colframe=Green!75!black,
    title={\centering Interfaces}]
    \begin{itemize}
        \item Ähnlich zu Typen-Aliasen, aber ausschließlich für Objekte
    \end{itemize}
    \begin{lstlisting}
        interface Espressomaschine {
            marke: string,
            modell: string,
            serienNr: number
        };

        const meinKA: Espressomaschine {
            marke: "De'Longhi",
            modell: "La Specialista Maestro",
            serienNr: 39472833
        };
    \end{lstlisting}
    \begin{itemize}
        \item Können um weitere Eigenschaften erweitert werden
    \end{itemize}
    \begin{lstlisting}
        interface Siebtraeger extends Espressomaschine {
            filterDurchmesser: number,
            kaffeeMuehle: boolean
        };

        const meinKA: Siebtraeger {
            marke: "De'Longhi",
            modell: "La Specialista Maestro",
            serienNr: 39472833,
            filterDurchmesser: 51
        };
    \end{lstlisting}
    \end{tcolorbox}

    \begin{tcolorbox}[
    colback=Yellow!5!white,
    colframe=Yellow!75!black,
    title={\centering Vereinigungen}]
    \begin{itemize}
        \item Können Variablen mehr als einen Typ haben, spricht man von Vereinigungen (unions)
        \item Zulässige Typen werden mit einem Oder-Operator \textbf{|} voneinander getrennt
    \end{itemize}
    \begin{lstlisting}
        let switch: number | boolean = 1;
        switch = false; // korrekt!
        switch = 'an'; // Fehler!
    \end{lstlisting}
    \end{tcolorbox}

    \begin{tcolorbox}[
    colback=Orange!5!white,
    colframe=Orange!75!black,
    title={\centering Funktionen}]
    \begin{itemize}
        \item Der Typ der einzelnen Parameter und der Typ des Rückgabeparameters können festgelegt werden
        \item Werden keine Typen festgelegt, wird \textbf{any} angenommen
        \item Eine Rückgabetyp \textbf{void} bedeutet, dass die Funktion keinen Wert zurückgibt
    \end{itemize}
    \begin{lstlisting}
        function Hallo(): void {
            alert('Hallo!');
        }

        function Max(wert1: number, wert2: number): number {
            return wert1 > wert2 ? wert1 : wert2;
        }
    \end{lstlisting}
    \begin{itemize}
        \item Optionale Parameter werden mit einem ? hinter dem Namen gekennzeichnet
        \item Optionale Parameter können nur am Ende stehen
        \item Default-Parameter sind ebenfalls optional, haben in diesem Fall jedoch einen fest vorgegebenen Wert
        \item Dieser wird über eine Zuweisung im Funktionskopf festgelegt
    \end{itemize}
    \begin{lstlisting}
        function Summe(wert1: number, wert2: number, wert3?: number)
        : number {
            return wert1 + wert2 + (0 || wert3);
        }

        function Wurzel(a: number, n: number = 2): number {
            return a ** (1/n);
        }
    \end{lstlisting}
    \end{tcolorbox}

    \begin{tcolorbox}[
    colback=Red!5!white,
    colframe=Red!75!black,
    title={\centering Casting}]
    \begin{itemize}
        \item Typ einer Variable explizit umdeuten
        \item Variable wird \textbf{as} gefolgt vom gewünschten Typ nachgestellt
        \item Funktioniert nur, wenn Inhalt der Variablen dies auch erlaubt
    \end{itemize}
    \begin{lstlisting}
        let a: unknown = 'Hallo!';
        alert(a as string);
    \end{lstlisting}
    \end{tcolorbox}

    \begin{tcolorbox}[
    colback=Orange!5!white,
    colframe=Orange!75!black,
    title={\centering Klassen in TypeScript}]
    \begin{itemize}
        \item Eigenschaften haben wie bei Objekten Typen
        \item Eigenschaften und Funktionen können individuelle Zugriffsrechte definieren
        \item \textbf{public} - (Standard) erlaubt den Zugriff von außerhalb der Klasse
        \item \textbf{private} - erlaubt nur den Zugriff aus Funktionen der Klasse
        \item \textbf{protected} - erlaubt den Zugriff von innerhalb und erweiterten Klassen
    \end{itemize}
    \begin{lstlisting}
        class KaffeeAutomat {
            public marke: string;
            public farbe: string;
            private an: boolean;
            constructor(kAMarke, kAFarbe, kAAn) { ... };
            public anschalten = function(): void {
                an = true;
            }
        }
    \end{lstlisting}
    \end{tcolorbox}

    \begin{tcolorbox}[
    colback=Yellow!5!white,
    colframe=Yellow!75!black,
    title={\centering Vergleich TypeScript - JavaScript}]
    \begin{itemize}
        \item TypeScript
        \begin{itemize}
            \item Starke Typisierung möglich
            \item Fehlerhafte Verwendung führt zu Fehlermeldung
            \item Array-Elemente haben gleichen Typ
            \item Tupel als typisierte Arrays
            \item Objekte mit typisierten Eigenschaften
            \item Funktionen haben typisierte Parameter und Rückgabewerte
            \item Klassen erlauben moderne objekt-orientierte Programmierung
        \end{itemize}
        \item JavaScript
        \begin{itemize}
            \item Lose Typisierung
            \item Fehlerhafte Verwendung wird nicht unmittelbar erkannt
            \item Array-Elemente können individuelle Typen haben
            \item Eigenschaften von Objekten haben beliebige Typen
            \item Parameter und Rückgabewerte von Funktionen haben keine Typen
            \item Klassen sind nahezu identisch zu Konstruktoren ohne Klasse
        \end{itemize}
    \end{itemize}
    \end{tcolorbox}

\end{document}
